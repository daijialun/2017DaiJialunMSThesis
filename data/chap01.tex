\chapter{绪论}
\label{cha:intro}

\section{课题的研究背景及意义}
海洋涵盖了地球表面超过70\%的面积,包括大约13.5亿立方千米容量的水资源,占据地球上总水量的97\%,而地球的生命起源于海洋,海洋孕育了地球上大部分的动物和植物,对于地球上各式各样的生命起着不可忽视的作用;并且海洋通过对热量的吸收和大气的传递,维持着全球气候的稳定。因此海洋中丰富的资源对天气的作用和对人类的生活具有深远的影响。

浮游生物是生活在海洋、湖泊以及河川等水域中的生物,由于其不具有大范围移动的能力,平常都是漂浮在水面上,其中海洋浮游生物占据了很大部分。海洋浮游生物可以分为浮游动物和浮游植物两大类。浮游生物的覆盖范围非常广,从小型的细菌、病毒,到大型的水母等都包含在浮游生物的范围内,因此可以通过体型的大小来区分浮游生物,不过总体而言,浮游生物的体型偏小,大多数浮游生物需要通过显微镜等仪器进行观察。

浮游生物是一个庞大而复杂的生态类群,其在海洋食物链中起着最基础与最关键的作用,保持与维护了海洋食物链的平衡与稳定。浮游生物根据生物的体型大小,可将其分为分为多个不同种群:小型浮游种群如纤毛虫等,中型生物种群如小型挠足类等,较大的浮游种群如水母等~\cite{2003}。

浮游生物种类繁多,数量庞大,是海洋生物的主要成员,其研究的开展可以促进渔业生产和海洋科学研究的不断进步。海洋中的洋流流向情况、地理环境以及海洋盆地地形,都会对浮游生物的物种多样性和丰富度造成不同程度的影响。因此不同情景下的浮游生物物种分布等情况都不相同。对浮游生物的调查,包括物种组成、丰富度分布、以及其空间等变化,在海洋生态系统、环境检测与海洋渔业等领域具有科学性与实践性的意义。

另外,浮游动物的大量减少会对全球生态系统造成毁灭性的影响。反之,浮游动物的暴增也会对全球生态系统带来巨大灾难。浮游动物的分布以及丰富程度影响着海洋生态系统的平衡。因此,对浮游动物的种类的丰富度和分布的检测是非常重要的工作。通过对浮游动物的监测与获取信息的处理,可分析出某片海域的浮游生物物种分布与推断该海域相应信息,在某些特殊场合,快速得到合理的解决方案。其中,对浮游动物图像的分类识别是关键技术。 

早期的浮游生物图像识别工作,主要是依靠海洋领域或者浮游生物领域相关的研究人员与专家,人工地对浮游生物图像进行相关的采样、检测与识别。人工对浮游生物识别必须依靠具有海洋浮游生物相关专业知识与经验的专家或研究学者,这些研究学者与专家需要经过大量的工作经验与知识才能够掌握海洋浮游生物识别技能,并且这些专家的人数相对于其他生物领域的研究人员,在人力方面资源较少,而且人为的误差等也一定程度地降低了图像分类准确率。

另外,在当今时代如果通过人工对浮游生物进行识别的话,是一个非常消耗时间的过程,甚至可能会出现,在一天之内采集的浮游生物图像,要花费一年或者更长的时间来识别与分析它们。因此对于浮游生物图像的分类,如果依靠专家的人工分类,将需要耗费大量的时间与人力资源,并且可能会一定程度造成准确率的降低,无法应用在有大规模浮游生物图像数据或要求浮游生物图像高准确率分类的情况下,而普通的工作人员更是无法完成这项困难的任务,所以当今情形下迫切地需要一项技术或方法,可以有效快速地解决浮游生物图像识别的问题。如今国内有不少科研人员正在研究有效的浮游生物图像自动分类技术。

如果通过有效的海洋浮游生物图像分类技术,可保证对收集到的大量浮游生物图像实现准确的识别分类。随后进行的种类识别与统计等工作,可进一步实现浮游生物种类组成和丰富度分布的分析,节约了大量的人力与时间成本,而且对海洋生态环境以及环境监测具有重要意义。

\section{国内外研究现状}

由于近年来浮游生物图像数据集的数量持续增长,因此浮游生物图像分析已经得到越来越多的关注。海外的研究人员较早地展开了对浮游生物图像的相关识 别研究,而且已取得了较好的成果。

较早时期对浮游生物图像的研究方法主要是使用特征提取与分类器结合的方法实现的。Xiaoou Tang提出了一个模式识别系统来实现对拖拽式水下视频显微系统实时采集到的大量浮游生物图像进行分类~\cite{tang1998automatic}。这个方法主要将灰度形态学的颗粒测定法与传统的不变矩特征和傅里叶边缘描述子相结合,由此生成浮游生物的形状与纹理信息的特征矩阵。在学习矢量量化网络分类器的辅助下,其方法对于6类浮游动物图像分类任务的准确率,与一个受过训练的生物学家人工识别这些图像所达到的准确率相当。但是该方法只能实现少类别的浮游生物图像分类,不能应用在大规模的图像分类问题上;而且所采取的特征主要是局部特征,没有综合考虑到其他形态学上的重要特征。

欧美国家的藻类研究专家在研究关于藻类图像的自动识别相关技术方面,已经取得了一定的成果。ADIAC(Automatic Diatiom Identification And Classfication)~\cite{du2002automatic}是使用浮游生物轮廓特征和条纹特征对藻类图像进行识别的。此系统基于轮廓特征和条纹特征的结合,应用与多种类硅藻的识别中,识别率达到 85\%左右。但是其在最近邻分类方法的基础上仅依靠四种曲率特征,只适用于部分特定的藻种图像,不能适应常见的各种情况。

欧盟共同体研究项目DiCANN (Dinflagellate Categorisation by Artifiial Neural Network)~\cite{culverhouse2003experts}的研究人员对4种浮游植物和23种浮游动物进行实验,利用人工神经 网络所搭建的系统对这些浮游生物进行自动分类识别。该方法中使用了多种特征 提取和转换方法:离散傅里叶变换、二阶统计量、Sobel算子、统计直方图和Gabor 小波变换等算法,提取了浮游生物形状特征和表面纹理灰度特征。随后在神经网 络中对所提取的多种特征进行训练学习,取得了84\%的准确率,在当时与专家人工的识别率相当,但是效率却有极大的提升。

相似地,Jalba等人~\cite{jalba2005automatic}等人也把侧重点放在形态学特征上,其将藻类图像中物体的轮廓特征和曲率特征提取出来,并且基于决策树和最近邻两种分类器进行组合实验,可达到90\%的准确率,但是该方法只能识别该论文中的藻类图像,对于该种类图像效果较好,然而对于其他种类的藻类图像不能很好分类识别,存在较大的局限性。 

Yang和Chou~\cite{yang2005comparative}使用了最近邻分类器对浮游动物的轮廓和不变矩等特征进行了分析描述,并且对最近邻分类器使用了不同方法的改进和相应的实验结果分析。虽然实验结果中最高的识别率可达到95\%,但是与之前方法存在的问题相似,所使用的特征不够具有代表性,且该方法所训练的图像只针对所训练的浮游动物图像,应用泛化性不强。 

而对于基于图像的流式细胞仪所采集的浮游植物图像,Heidi Sosik和Robert Olso~\cite{sosik2007automated}对其收集的22类浮游植物使用了自动系统分类的方法,完成对浮游植物分类的任务。其方法所提取的特征不仅包含了形状纹理等基本特征,而且添加了方向不变矩与共生矩阵的统计特性等额外信息。最后通过特征选择算法与SVM分类器的组合,实现对浮游植物图像的识别。

Gaby Gorsk~\cite{gorsky2010digital}在使用ZooScan集成系统的基础上,采集浮游生物图像与提取超过 60 种浮游生物的形态特性作为特征,训练6种不同的分类器(NN、SVM、Random Forest等)实现对20类浮游生物图像的预测。然而,60种浮游生物的简单形态特征(主要包括长,宽,周长与灰度等特征)相比计算机视觉方法所提取的特征,只是实现了数量上的增加,但是特征的表达能力上却没有显著的提升。 



而对于国内关于海洋浮游生物图像的自动识别研究目前处于起步阶段,且由于国内相关的数字图像处理与计算机视觉技术等领域发展较慢,因此国内的浮游生物图像自动识别技术仍然有很多问题需要解决。

天津大学的王明时等人\cite{wangmingshi2004}提出了使用形态学方法解决赤潮藻图像识别分类问题,根据浮游植物中藻类的轮廓、圆度、矩形度和扁度等特征,通过树状判别算法进行识别,得到了较好的结果。但是由于该方法中主要研究的浮游藻类呈现圆形或椭圆形等规则均匀的形状,因此对于实际海洋环境中外形复杂的浮游生物而言,所能取得效果有限,需要进一步提升其结果。

徐雷等研究学者\cite{xulei2003}使用数字图像处理方法,将浮游植物中的夜光藻和环境中的杂质颗粒辨别开,但是只根据简单的几何特征,因此只能将体型大小差异较大的夜光藻和杂质颗粒区分开来。另外这个只是简单的对该场景下的浮游生物和环境杂质分离问题进行解决,不能应用于相关的浮游生物实际问题中。 

来自厦门大学的王博亮等人\cite{zheng2009}所研发的关于流式细胞计数的赤潮藻实时检测系统,组成了一套高速图像采集和分析系统,从系统所采集到的浮游生物图像中,提取了包括傅里叶描述子和不变矩等多种特征,利用多级主特征向量评估算法减少了特征维度,随后通过最近邻分类器对这些常见的赤潮藻种实现分类。虽然该方法中对赤潮藻类的种类识别不多,但是样本基数较多,而且准确率能够达到90\%以上。 

杨晨辉等人\cite{chencheng2009}以模糊算子和边缘检测算法,采用简单的几何特征和矩形特征矩, 并且采用了灰度共生矩阵衍生的条纹特征,最后使用树状判别方法设计纹理分类 器,实现了浮游植物图像分类。但是该方法受训练样本数量等影响较大,随着这 些因素的改变,识别率在80\%到85\%之间浮动,因此还需要进一步保持方法的稳定性。

中国海洋大学的姬光荣等\cite{qiaoxiaoyan2010}从多种常见赤潮藻种出发,研制了基于赤潮藻的显微图像分类算法系统,针对包含角毛的赤潮藻种,提出了基于灰度曲面方向的算法,通过提取完整的角毛藻细胞,并利用其形态特征,根据与基本几何特征不同的骨架树拓扑结构,保证算法中基础特征的有效提取。

虽然国内关于浮游生物相关图像的识别研究技术的发展不快,但是仍然取得了一定成果。不过这些方法依然具有很大的局限性,需要对这些方法进行改进提升,才能保证更大的应用潜力。

近些年来,深度学习技术的出现,在大规模的图像分类与识别领域,超越了之前所使用的方法,在大规模图像数据集的基础上,可实现高准确率的图像分类与识别,是一个新兴而有效的识别方法。因此,有一些学者与研究人员已经将深度学习技术应用在浮游生物的识别领域中,用来解决浮游生物的检测与分类问题。

\section{课题来源}

国家自然科学基金项目“基于视觉注意结合生物形态特征的海洋浮游植物显微图像分析”(批准号:61301240)、国家自然科学基金项目“基于生物形态特征的中国海常见有害赤潮藻显微图像识别”(批准号:61271406)和中央高校基本科研业务费项目“海洋浮游动物原位探测与分析系统”(批准号:201562023)。


\section{论文组织结构安排}

本文在总结了目前国内外对浮游生物图像分类方法的基础上,将当今在解决图像分类问题上已经取得优异效果的深度学习方法,用于解决浮游生物图像分类问题,并且对浮游生物图像分类的深度学习方法进行了深层次的研究,提出了一种基于多特征卷积神经网络的浮游生物图像分类研究。

第一章 为绪论部分,主要介绍了浮游生物的相关背景情况,在海洋生态环境中的重要性,对浮游生物研究的意义,以及目前国内外对浮游生物图像分类的相关研究进展。

第二章 主要讨论了深度学习方法的相关背景,具体算法以及应用场景。简单地说明了作为深度学习基础的人工神经网络相关知识,以及卷积神经网络的历史发展与技术原理。

第三章 简要介绍了关于浮游生物的数据集,以及探讨了卷积神经网络在浮游生物图像分类任务的可行性和影响因素。

第四章 主要说明了本论文所提出的基于多特征卷积神经网络的浮游生物图像分类方法的具体实现步骤。

第五章 进行了本论文所提出的多特征卷积神经网络在浮游生物图像数据集上的具体实验结果。

第六章 对全本所做的工作内容和具体贡献进行总结与讨论,分析所提出方法中的不足之处,对将来的工作内容进行了展望。
