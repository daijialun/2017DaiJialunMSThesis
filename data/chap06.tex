\chapter{总结与展望}

\section{总结}

浮游生物在海洋生物圈和全球生态系统中起着重要作用,是海洋生物链的最底层,保证海洋生物圈的稳定,并且维持着全球生态系统的平衡。因此,对浮游生物的相关研究吸引了越来越多的关注。浮游生物图像的自动识别是一项非常重要的技术,其中就包括了浮游生物的自动分类。

传统的图像分类方法,主要是通过人工特征提取和分类器设计所完成的。但是由于浮游生物图像的特点,传统的图像分类方法并不适用于浮游生物图像的自动分类。基于深度学习的卷积神经网络,适用于解决大规模、复杂、困难的图像分类问题,与传统图像分类方法相比,能够挖掘出更抽象、更高维度的特征,
识别准确率更高,能更有效地解决问题。


本论文讨论了深度学习的基础理论和研究了多特征卷积神经网络的融合模型,进行了相关的探索和实验,主要研究工作如下:

\begin{enumerate}
\item 分析浮游生物图像特点,总结出浮游生物图像中存在的类间相似性和类内差异性问题,即不同种类的浮游生物,形状可能相似,纹理基本不同;而相同种类的浮游生物,形状可能不同,纹理基本相同。随后提出了形状和纹理相结合的浮游生物图像分析方法。
\item 将深度学习技术应用在多类别的浮游生物分类问题上,探究卷积神经网络在浮游生物图像分类问题的可行性。从自底向上的角度出发,根据网络深度、卷积数量、数据增强等方面对网络结构进行分析和改进。实验结果验证了卷积神经网络在解决浮游生物图像分类问题的有效性。
\item 从生物形态学与计算机视觉的角度,将浮游生物图像进行不同形式的变换,提出了一种基于多特征卷积神经网络的浮游生物图像分类方法。该方法从全部特征、形状特征和纹理特征出发,在多特征卷积网络模型中进行独立转换、训练和融合。该模型在多个不同维度的深层卷积网络基础上,对浮游生物图像进行分析处理和特征融合,将得到更具体和更高维度的特征,提升了网络模型的学习和表达能力。实验结果表明了该方法可以提升准确率,有效地解决浮游生物分类问题。
\end{enumerate}


\section{展望}

本论文提出了一种基于多特征卷积神经网络模型用于解决浮游生物图像分类问题,尽管该方法在实验中取得了较好的结果,但是仍然存在着不足之处需要在将来的工作继续改进:

\begin{enumerate}
\item 目前该基于多特征卷积神经网络的浮游生物图像分类方法是使用30类浮游生物图像数据集进行训练的,但是实际场景下,浮游生物种类可能超过100类以上,因此后续还需要收集更多数据对该方法进行训练测试,不断对其改进,保证该方法的实际可用性。
\item 该方法在提取全局特征图像的过程中,由于只是采用最简单的图像处理方法,所以在转换全局特征方面有所影响。在更高效的提取浮游生物形状信息的前提下,该方法可会得到更好的结果。
\item 由于该方法在训练过程中,需要对图像进行特征变换,而且网络模型有三个子网络,网络参数较多,另外全连接交叉也增加了全连接层的参数数量,一定程度上导致了训练时间增长和训练难度加大的问题。如果能减少训练时间,保证网络有效训练,那么就能提升网络模型的推广型和实用性。
\end{enumerate}
